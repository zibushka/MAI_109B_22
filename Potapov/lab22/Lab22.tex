\documentclass[a5paper, 16pt]{book}

\usepackage[left=10mm, top=10mm, right=20mm, bottom=10mm, nohead, nofoot]{geometry}
\usepackage[english, russian]{babel}
\usepackage[utf8]{inputenc}
\usepackage{amsfonts}
\usepackage{amsmath}

\setlength{\headheight}{0mm}
\setlength{\headsep}{0mm}
\setcounter{page}{285}

\date{09.03.2023}
\author{Егор Потапов}

\begin{document}
    \begin{center}
        $\S7.$ \textsc{ПЕРВООБРАЗНАЯ}
    \end{center}
    \par\textsc{Пример 2.} Функция $F(x) = arcctg \, \frac{1}{x}$ является первообразной для функции $f(x) = \frac{1}{1 + x ^ {2}}$ как на промежутке всех положительных чисел, так и на полуоси отрицательных чисел, ибо при $x \neq 0$
    $$F^{'} (x) = \textbf{-} \frac{1}{1 + \big(\frac{1}{x} \big)^{2}} \cdot \bigg( \textbf{-} \frac{1}{x ^ {2}} \bigg) = \frac{1}{1 + x ^ {2}} = f(x) .$$
    \parКак обстоит дело с существованием первообразной и каково множество первообрахных данной функции?
    \parВ интегральном исчислении будет доказан фундаментальный факт о том, что любая непрерывная на промежутке функция имеет на этом промежутке первообразную.
    \parМы приводим этот факт для информации читателя, а в этом параграфе используется, по существу, лишь следующая, уже известная нам (см. гл. V, $\S3,$ п.1) характеристика множества первообразных данной функции на числовом промежутке, полученная из теоремы Лагранжа.
    \par\textsc{Утверждение 1.} \textit{Если $F_1(x)$ и $F_2(x)$ \textbf{"---} две первообразные функции $f(x)$ на одном и том же промежутке, то их разность $F_1(x) - F_2(x)$ постоянна на этом промежутке.}
    \parУсловие, что сравнение $F_1$ и $F_2$ ведется на связном промежутке, как отмечалось при доказательстве этого утверждения, весьма существенно. Это можно заметить также из сопоставления примеров 1 и 2, в которых производные функций $F_1(x) = arctg \, x$ и $F_2(x) = arcctg \, \frac{1}{x}$ совпадают в области $\mathbb{R} \; \backslash \; 0$ их совместного определения. Однако
    $$F_1(x) - F_2(x) = arctg \, x - arcctg \, \frac{1}{x} = arctg \, x - arctg \, x = 0 ,$$
    если $x > 0 ,$ в то время как $F_1(x) - F_2(x) \equiv \textbf{-} \pi$ при $x < 0$, ибо при $x < 0$ имеем $arcctg \, \frac{1}{x} = \pi + arctg \, x .$
    \parКак и операция взятия дифференциала, имеющая свое название <<дифференцирование>> и свой математический символ $dF(x) = F ^ {'} (x) \, dx$, операция перехода к первообразной имеет свое название <<неопределенное интегрирование>> и свой математический символ
    \begin{flushright}
        $\int f(x) \, dx, \qquad \qquad \qquad \qquad \qquad \qquad \textrm{(1)}$
    \end{flushright}
    называемый \textit{неопределенным интегралом} от функции $f(x)$ на заданном промежутке.
    \parТаким образом, символ (1) мы будем понимать как обозначение любой из первообразных функции $f$ на рассматриваемом промежутке.
    \parВ символе (1) знак $\int$ называется знаком \textit{неопределенного интеграла}, $f \textbf{"---}$ \textit{подынтегральная функция} , а $f(x) \, dx \textbf{"---}$ \textit{подынтегральное выражение}.
    \parИз утверждения 1 следует, что если $F(x) \textbf{"---}$ какая-то конкретная пер-

    \newpage

    \begin{center}
        \textsc{ГЛ. V. ДИФФЕРЕНЦИАЛЬНОЕ ИСЧИСЛЕНИЕ}
    \end{center}
    вообразная функции $f(x)$ на промежутке, то на этом промежутке
    \begin{flushright}
        $\int f(x) \, dx = F(x) + C, \qquad \qquad \qquad \qquad \qquad \qquad \textrm{(2)}$
    \end{flushright}
    т.е. любая другая первообразная может быть получена из конкретной $F(x)$ добавлением некоторой постоянной.
    \parЕсли $F ^ {'} (x) = f(x)$, т.е. $F \textbf{"---}$ первообразная для $f$ на некотором промежутке, то из (2) имеем
    \begin{flushright}
        $\fbox{$d\, \int f(x) \, dx = dF(x) = F ^ {'} \, dx = f(x) \, dx.$} \qquad \qquad \qquad \textrm{(3)}$
    \end{flushright}
    \parКроме того, в соответствии с понятием неопределенного интеграла как любой из первообразных, из (2) следует так же, что
    \begin{flushright}
        $\fbox{$\int dF(x) = \int F ^ {'} (x) \, dx = F(x) + C.$} \quad \qquad \qquad \qquad \textrm{(4)}$
    \end{flushright}
    \parФормулы (3) и (4) устанавливают взаимность операций дифференцирования и неопределенного интегрирования. Эти операции взаимно обратны с точностью до появляющейся в формуле (4) неопределенной постоянной $C$.
    \parДо сих пор мы обсуждали лишь математическую природу постоянной $C$ в формуле (2). Укажем теперь ее физический смысл на простейшем примере. Пусть точка движется по прямой так, что ее скорость $\upsilon (t)$ известна как функция времени (например, $\upsilon (t) \equiv \upsilon$). Если $x(t) \textbf{"---}$координата точки в момент $t$б то функция $x(t)$ удовлетворяет уравнению $\dot x (t) = \upsilon (t)$, т.е. является первообразной для $\upsilon (t)$. Можно ли по скорости $\upsilon (t)$ в каком-то интервале времени восстановить положение точки на оси? Ясно, что нет. По скорости и промежутку времени можно определить величину пройденного за это время пути $s$, но не положение на оси. Однако это положение также будет полностью определено, если указать его хотя бы в какой-то момент, например при $t = 0$, т.е. задать начальное условие $x(0) = x_0$. До задания начального условия закон движения $x(t)$ мог быть любым среди законов вида $x(t) = \tilde x (t) + c$, где $\tilde x (t) \textbf{"---}$ любая конкретная первообразная функции $\upsilon (t)$, а $c \textbf{"---}$ произвольная постоянная. Но после задания начального условия $x(0) = x_0$ вся неопределенность исчезает, ибо мы должны иметь $x(0) = \tilde x (0) + c = x_0$, т.е. $c = x_0 - \tilde x (0)$, и $x(t) = x_0 + [\tilde x (t) - \tilde x (0)]$. Последняя формула вполне физична, поскольку произвольная первообразная $\tilde x$ участвует в формуле только в виде разности, определяя пройденный путь или величину смещения от известной начальной метки $x(0) = x_0$.
    \par\textbf{2. Основные общие приемы отыскания первообразной.} В соответствии с определением символа (1) неопределенного интеграла, он обозначает функцию, производная которой равна подынтегральной функции. Исходя из этого определения, с учетом соотношения (2) и законов дифференцирования можно утверждать, что справедливы следующие соотношения:

    \newpage

    \begin{center}
        $\S7.$ \textsc{ПЕРВООБРАЗНАЯ}
    \end{center}
    \begin{flushleft}
        $$\textrm{a.} \qquad \int (\alpha u(x) + \beta \upsilon (x)) \, dx = \alpha \int u(x) \, dx + \beta \int \upsilon (x) \, dx + c. \quad \qquad \, \,(5)$$
        $$\textrm{b.} \qquad \int (u \upsilon) ^ {'} (x) \, dx = \int u ^ {'} (x) \upsilon (x) \, dx + \int u(x) \upsilon ^ {'} (x) \, dx + c. \quad \qquad \quad (6)$$
        \quad \textrm{c.}\textit{Если на некотором промежутке I_x}
    \end{flushleft}
    $$\int f(x) \, dx = F(x) +c,$$
    a $\varphi : I_t \to I_x \textbf{"---}$ \textit{гладкое (т.е. непрерывно дифференцируемое) отображение промежутка I_t в I_x, то}
    \begin{flushright}
    $\int (f \circ \varphi)(t) \varphi ^ {'} (t) \, dt = (F \circ \varphi)(t) + c. \qquad \qquad \qquad (7)$
    \end{flushright}
    \parРавенства (5), (6), (7) проверяются прямым дифференцированием их левой и правой частей с использованием в (5) линейности дифференцирования, в (6) правила дифференцирования произведения и в (7) правила дифференцирования композиции функций.
    \parПодобно правилам дифференцирования, позволяющим дифференцировать линейные комбинации, произведения и композиции уже известных функций, соотношения (5), (6), (7), как мы увидим, позволяют в ряде случаев сводить отыскание первообразной данной функции либо к построению первообразных более простых функций, либо вообще к уже известным первообразным. Набор таких известных первообразных может составить, например, следующая краткая таблица неопределенных интегралов, полученная переписыванием таблицы производных основных элементарных функций (см. $\S2$, п.3):
    $$\int x ^ {a} \, dx = \frac{1}{a + 1} x ^ {a + 1} + c \qquad (a \neq -1),$$
    $$\int \frac{1}{x} \, dx = ln \, |x| + c,$$
    $$\int a ^ {x} \, dx = \frac{1}{ln \, a} a ^ {x} + c \qquad (0 < a \neq 1),$$
    $$\int e ^ {x} \, dx = e ^ {x} + c,$$
    $$\int sin \, x \, dx = -cos \, x +c,$$
    $$\int cos \, x \, dx = sin \, x + c,$$
    $$\int \frac{1}{cos ^ {2} \, x} \, dx = tg \, x + c,$$
    $$\int \frac{1}{sin ^ {2} \, x} \, dx = -ctg \, x + c,$$
    \begin{equation*}
        \int \frac{1}{\sqrt{1 - x ^ {2}}} \, dx =
        \begin{cases}
            arcsin \, x +c, \\
            -arccos \, x + \tilde c,
        \end{cases}
    \end{equation*}
    
\end{document}