\documentclass[a5paper, 16pt]{book}

\usepackage[left=5mm, top=5mm, right=10mm, bottom=10mm, nohead, nofoot]{geometry}
\usepackage[english, russian]{babel}
\usepackage[utf8]{inputenc}

\setlength{\headheight}{0mm}
\setlength{\headsep}{0mm}
\setcounter{page}{22}

\date{21.03.2023}
\author{Никита Мозговой}

\begin{document}
    \begin{center}
        \begin{spacing}
            \textit{Гл. 1. Введение}
            \noindent\rule{\textwidth}{1pt}
		\end{spacing}
    \end{center}
    \begin{center}
        \textbf{\S4. Прогрессии. Суммирование. Бином Ньютона. Числовые неравенства.}
    \end{center}
    \par\texttt{СПРАВОЧНЫЕ СВЕДЕНИЯ}
    \par\textbf{1. Числовая последовательность.}
    \par1) Если каждому натуральному числу $n$ поставлено в соответствие некоторое действительное число $x_n$, то говорят, что задана \textit{числовая последовательность} (или просто последовательность)
    $$x_1, x_2, ..., x_n, ... $$
    Кратко последовательность обозначают символом ${x_n}$ или $(x_n)$, число $x_n$ называют \textit{членом} или \textit{элементом} этой последовательности, $n \, \textbf{---}$ номером члена $x_n$.
    \par2) Последовательность обычно задается либо формулой, с помощью которой можно вычислить каждый ее член по соответствующему номеру, либо формулой, позволяющей находить члены последовательности по известным предыдущим (рекуррентной формулой).
    \par\textbf{2. Арифметическая прогрессия.}
    \par1) \textit{Арифметическая прогрессия} $\textbf{---}$ последовательность ${a_n} \textbf{---}$ определяется рекуррентной формулой
    $$a_{n+1} = a_n + d,$$
    где $a_1$ и $d \textbf{---}$ заданные числа; число $d$ называется \textit{разностью} арифметической прогрессии.
    \par2) Формула $n$-го члена арифметической прогрессии:
    $$a_n = a_1 + d(n - 1),$$
    \par3) Каждый член арифметической прогрессии, начиная со второго, равен среднему арифметическому его соседних членов, т.е. при $k \geq 2$ справедливо равенство
    $$a_k = \frac{a_{k - 1} + a_{k + 1}}{2}.$$
    \par4) Сумма $n$ первых членов арифметической прогрессии выражается формулой
    $$S_n = \frac{a_1 + a_n}{2} \cdot n = \frac{2a_1 + d(n - 1)}{2} \cdot n.$$
    \par\textbf{3. Геометрическая прогрессия.}
    \par1) \textit{Геометрическая прогрессия} $\textbf{---}$ последовательность ${b_n} \textbf{---}$ определяемая рекуррентной формулой
    $$b_{n + 1} = b_n q$$
    где $b_1$ и $q \textbf{---}$ заданные числа, отличные от нуля; число $q$ называют \textit{знаменателем} геометрической прогрессии.
    \par2) Формула $n-$го члена геометрической прогрессии:
    $$b_n = b_1 q^{n - 1} .$$

    \newpage

    \setcounter{page}{48}
    \begin{center}
        \begin{spacing}
            \small{\textit{Гл. 1. Введение}}
            \noindent\rule{\textwidth}{1pt}
		\end{spacing}
    \end{center}
    \parВ частности, если $Q(x) = x - a$, где $a \, \textbf{---}$ заданное число ($a \in R$ или $a \in C$), а $P(x) = Q_n (x) \, \textbf{---}$ многочлен степени $n$, то в формуле (2) частное $T(x) = \widetilde Q _ {n - 1} (x) \, \textbf{---}$ многочлен степени $n - 1$, а $R(x) = r \, \textbf{---}$ некоторое число. Итак, формула деления многочлена $Q_n (x)$ степени $n$ на двучлен $x - a$ имеет вид
    \begin{flushright}
    $Q_n (x) = (x - a)Q_{n - 1} (x) + r. \qquad \qquad \qquad \qquad \qquad \qquad (3)$
    \end{flushright}
    \par5) \texttt{Теорема Безу.} \textit{Число a является корнем многочлена $Q_n (x) тогда и только тогда, когда этот многочлен делится без отсатка на x - a, т.е. справедливо равенство$}
    $$Q_n (x) = \widetilde Q _{n - 1} (x)(x - a).$$
    \par6) Числа $a$ называют \textit{корнем многочлена $Q_n (x) кратности k,$} если существует число $k \in N$ и многочлен $Q _{n - k} ^ * (x)$ такие, что для всех $x$ ($x \in R$ или $x \in C$) справедливо равенство  
    \begin{flushright}
    $Q_n (x) = (x- a)^k Q_{n - k} ^ * (x), \qquad \qquad \qquad \qquad \qquad \qquad (4)$
    \end{flushright}
    где
    \begin{flushright}
    $Q_{n - k} ^ *  \neq 0, \qquad \qquad \qquad \qquad \qquad \qquad \qquad \qquad (5)$
    \end{flushright}
    Если $a \in R$ и коэффициенты многочлена $Q_n (x) \, \textbf{---}$ действительные числа, то условия (4), (5) выполняются тогда и только тогда, когда
    $$Q_n (a) = 0, \, Q_n ^ {'} (a) = 0, ..., \, Q_n ^ (k - 1) (a) = 0, \, Q_n ^ {(k)} (a) \neq 0.$$
    \par7) Если $Q(x) = x^2 +px +q$, где $p \in R, q \in R, p^2 - 4q < 0,$ то корни $x_1$ и $x_2$ многочлена $Q(x) \, \textbf{---}$ комплексно сопряженные числа:
    $$x_1 = -\frac{p}{2} + i\sqrt{q - \frac{p^2}{4}} , \, \, x_2 = -\frac{p}{2} - i\sqrt{q - \frac{p^2}{4}} .$$
    \par8) Если $Q_n (x) \textbf{---}$ многочлен с действительными коэффициентами, а $x_0 = \gamma + i\delta, \delta \neq 0,$ то число $\overline{x} _0 = \gamma - i\delta$ также является корнем этого многочлена.
    \par9) Целые корни алгебраического уравения $Q_n (x) = 0$, где $Q_n (x) \, \textbf{---}$ многочлен с целыми коэффициентами, являются делителями свободного члена.
    \par\textbf{2. Разложение многочлена на множители.}
    \par1) \texttt{Теорема Гаусса} (основная теорема алгебры). \textit{Алгебраическое уравнение степени $n \geq 1$, т.е. уравнение $Q_n (x) = 0$, где $Q_n (x) \, \textbf{---}$ многочленг \text{(1)} степени n, с действительными или комплексными коэффициентами имеет n корней при условии, что каждый корень считается столько раз, какова его кратность.}
    \par2) Пусть $Q_n (x) \, \textbf{---}$ многочлен (1) степени $n$  сдействительными коэффициентами, $a_j (j = 1,2,...,k) \, \textbf{---}$ все действительные корни этого многочлена, $\alpha _j \, \textbf{---}$ кратность корня $a_j$. Тогда
    $$Q_n (x) = C_n (x - a_1) ^ {\alpha _1} ... (x - a_k) ^ {\alpha _k} R(x) ,$$
    
    \newpage

    \setcounter{page}{49}
    \begin{center}
        \begin{spacing}
            \small{\textit{\S6. Многочлены. Алгебраические уравнения. Рациональные дроби}}
            \noindent\rule{\textwidth}{1pt}
		\end{spacing}
    \end{center}
    где $R(x) \, \textbf{---}$ многочлен с действительными коэффициентами степени $t = n - \sum_{j = 1} ^ {k} a_j,$ не имеющий действительных корней. Если $t > 1$, то многочлен $R(x)$ далжен делиться на многочлен $x ^ {2} + px + q = (x - x_0)(x - \overline{x}_0)$, где $x_0 = \gamma + i\delta \, (\delta \neq 0) \textbf{---}$ комплексный корень многочлена $R(x).$
    \parПусть $x_j$ и $\overline{x}_j \textbf{---}$ пара комплексно сопряженных корней многочлена $R(x), \beta _j \textbf{---}$ кратность этих корней,
    $$x ^ {2} + p_j x + q_j = (x - x_j)(x - \overline{x}_j), p_j \in R, q_j \in R,$$
    $x_j, \overline{x}_j \quad (j = 1,2,...,s) \textbf{---}$ все пары комплексно сопряженных корней многочлена $R(x).$ Тогда
    $$Q_n (x) = C_n (x - a_1)^{\alpha_1} ... (x - a_k)^{\alpha_k} (x^{2} + p_1 x + q_1)^{\beta_i} ... (x^{2} + p_s x + q_s)^{\beta_s} , \quad (6)$$
    $$\sum_{j = 1} ^ k \alpha_k + 2\sum_{j = 1} ^ s \beta_j = n.$$
    \par\textbf{3. Разложение правильной рациональной дроби на элементарные.}
    \par1) Пусть $P_m (x)$ и $Q_n (x) \textbf{---}$ многочлены степени $m$ и $n$. Если $m < n$, то функцию $\Large{\frac{P_m (x)}{Q_n (x)}}$ называют \textit{правильной рациональной дробью}, а при $m \geq n \, \textbf{---}$ \textit{неправильной}.
    \par2) Если $T(x)$ частное, а $R(x) \, \textbf{---}$ остаток от деления многочлена $P_m (x)$ на многочлен $Q_n (x)$, то
    $$\frac{P_m (x)}{Q_n (x)} = T(x) + \frac{R(x)}{Q_n (x)} ,$$
    где либо $R(x) = 0$ (в случае, когда многочлен $P_m (x)$ нацело делится на многочлен $Q_n (x)$), либо $R(x) \neq 0$, а дробь $\frac{R(x)}{Q_n (x)}$ является правильной.
    \par3) Пусть $P_m (x)$ и $Q_n (x) \, \textbf{---}$ многочлены с действительными коэффициентами, $\frac{R(x)}{Q_n (x)} \, \textbf{---}$ правильная дробь, а число $a \, \textbf{---}$ действительный корень кратности $k$ многочлена $Q_n (x)$. Тогда существуют действительные числа $A_1, A_2, ..., A_k$ такие, что 
    $$\frac{P_m (x)}{Q_n (x)} = \frac{A_k}{(x - a) ^ k} + \frac{A_{k - 1}}{(x - a) ^ {k - 1}} + ... + \frac{A_1}{x - a} + \frac{P^* (x)}{Q_{n - k} ^ * (x)} ,$$
    где $P ^ * (x) \, \textbf{---}$ многочлен с действительными коэффициентами или нуль, $Q_{n - k} ^ * (x) \, \textbf{---}$ частное от деления $Q_n (x)$ на $(x - a) ^ k$ при $P ^ * (x) \not\equiv 0.$ Дробь $\frac{P^* (x)}{Q_{n - k} ^ * (x)}$ является правильной, а числа $A_j (j = 1,2,...,k)$ и многочлен $P^* (x)$ определяется однозначно.
    
\end{document}